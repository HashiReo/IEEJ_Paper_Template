\documentclass[a4paper,10pt]{jsarticle}

% ========= 余白設定(電気学会の指定に準拠) =========
\usepackage[top=30mm,bottom=27mm,left=18mm,right=18mm]{geometry}
\setlength{\columnsep}{7mm} % カラム間7mm

% ========= 必要パッケージ =========
\usepackage[dvipdfmx]{graphicx}
\usepackage{amsmath,amssymb}
\usepackage{cite}
\usepackage{url}
\usepackage{multicol}

\pagestyle{empty} % ページ番号を非表示

\begin{document}

% ===========================================
% タイトル(1カラム幅で中央揃え)
% ===========================================
% 講演番号スペース(左上)
\noindent\hspace*{20mm}\par\vspace{2mm}

\begin{center}
  {\large 強化学習を用いた蓄電池制御アルゴリズムでの\\
  物理的制約に基づいた報酬設計}\\[2mm]

  橋場 怜央*,小平 大輔(筑波大学)\\[2mm]

  {\normalsize
  Proposal of Reward Design Based on Physical Constraints in\\
  a Reinforcement Learning-Based Battery Control Algorithm\\
  Reo Hashiba, Daisuke Kodaira (University of Tsukuba)
  }
\end{center}

\vspace{3mm}

\begin{abstract}
ここに概要を書きます。
\end{abstract}
 % ← abstract.tex を読み込む

\vspace{5mm}

% ===========================================
% ここから本文2カラム
% ===========================================
\begin{multicols}{2}

\section{はじめに}
ここに本文を書きます。

\section{DRLモデル}

\subsection{環境}
ここに本文を書きます。

\subsection{行動}
ここに本文を書きます。

\subsection{制約と報酬設計}
ここに本文を書きます。

\section{学習結果}
ここに本文を書きます。

\section{まとめ}
ここに本文を書きます。

\bibliographystyle{junsrt}
\bibliography{references}

\end{multicols}

\end{document}
